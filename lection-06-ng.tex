\documentclass[
 %reprint,
 aps, pra,
 amsmath,amssymb,
 11pt,
 final,
% notitlepage,
tightenlines,
 twoside,
 twocolumn,
 nofloats,
% nobibnotes,
nofootinbib,
 superscriptaddress,
%noshowpacs,
showkeys,
showkeywords,
%centertags
 ]
{revtex4-2}

\usepackage{fontenc}
\usepackage{graphicx}%Вставка картинок правильная
\usepackage{float}%"Плавающие" картинки
\usepackage{wrapfig}%Обтекание фигур (таблиц, картинок и прочего)
\usepackage{inputenc}
% \usepackage{amsmath, amssymb, mathrsfs}
\usepackage[T2A]{fontenc}
\usepackage[utf8x]{inputenc}
\usepackage[russian,english]{babel}
\usepackage{graphicx}% Include figure files
\usepackage{dcolumn}% Align table columns on decimal point
\usepackage{bm}% bold math
%\usepackage{longtable}

\input{maik.rty}

%\renewcommand{\rmdefault}{lh}
\setcitestyle{authoryear,round}
\setlength{\bibhang}{1.5em}

%Подключение некоторых команд из файла sao_cmd_author.tex
\input{sao_cmd_author.tex}

% Начало документа
\begin{document}

\selectlanguage{russian}

\title{Теория групп, 2025 г.}

\author{\firstname{Н.}~\surname{Галимуллин}}

\begin{abstract}
\textit{Введение.} В конспекте представлена часть лекции, на которой были обсуждены свойства подгрупп свободных групп, а так же алгебры Ли и смежные к ним определения.
\end{abstract}

\maketitle

{

\begin{center}
{\sc 1. Свободные группы.}
\end{center}

{\bf Определение.} Группа $F$ называется {\it свободной} группой ранга $n$, если существует множество $S = \{a_i\}^n_{i=1},$ где все элементы различны, такое, что для любого множества $\{\alpha_i\}^n_{i=1}$, если $a^{\alpha_1}_1\ldots a^{\alpha_n}_n = 1,$ то для всех $i$ $$a^{\alpha_i}_i=1.$$

\begin{center}
{\sc 1.1 Лемма Шрайера.}
\end{center}

{\bf Лемма Шрайера.} Пусть $F$ --- свободная конечнопорожденная группа. Тогда $H$ --- ее подгруппа конечного индекса --- свободна. \\

{\bf Формула Нильсона-Шрайера.} В указанных обозначениях выполнено соотношение 
$$
\text{rank} \ H = [H:F] \cdot (\text{rank} \ F - 1) + 1. 
$$

{ \bf Утверждение. } Пусть $F$ --- свободная группа ранга $n$, тогда 
$$
F_{ab} = F / [F, F] \cong \mathbb{Z}^n.
$$ 



\hrulefill

\begin{center}
{\sc 1.2. Группа соотношений.}
\end{center}

Пусть $G$ --- произвольная группа, $F$ --- свободная группа, которая вкладывается в $G$ субъективным гомоморфизмом $r$, с ядром $R,$ которое мы будем называть группой соотношений. В этом случае имеет место короткая свободная последовательность
$$
R \twoheadrightarrow F \hookrightarrow_r G.
$$

{ \it Замечание. } Тогда $R$ имеет конечного индекса при $|G| < \infty$, поскольку $|G| = [F : R].$ \\

{ \bf Утверждение.} Пусть $R_{ab} = R/[R, R]$. Отображение $G \times R_{ab} \to R_{ab}$
$$
(g, x[R, R]) \mapsto g' xg'^{-1}[R, R], 
$$
где $g' \in n^{-1}(g)$ определено корректно.\\

% {\it Доказательство.} Достаточно показать, что 
% $$
% (g'^{-1}g'')x(g'^{-1}g'')^-1 \equiv x \pmod{[R, R]}
% $$
% что эквивалетно 
% $$
% r x r^{-1}[R, R] = x [R, R]
% $$

% Вспоним, что $R$ является нормальной подгруппой, в частности $xrx^{-1} \in R$

% То есть $xrx^{-1} \in R$ но тогда $[r, x] \in [R, R]$.

% { \it Следствие. } Отображение задает структуру $G$-модуля на $R_{ab}.$

{ \it Замечание.} С помощью леммы Шрайера несложно убедиться, что

\[
R_{ab} \cong \mathbb{Z}^{1 + |G|(\text{rank } F-1)}
\]

На практике щупать не нужно, используется в топологической алгебре.

\hrulefill

\begin{center}
{\sc 2. Алгебры Ли.}
\end{center}

{ \bf Определение.} Алгеброй Ли ${\displaystyle {\mathfrak {L}}}$ 
 называют $A$ --- $R$-модуль над полем $K$, c определенной на на нем билинейным отображением $$[ ., . ] : A \times A \to A,$$
которое удовлетворяет 
\begin{enumerate}
    \item Условию антисимметричности 
    $$
    [x, y] = - [y, x]
    $$

    \item { Тождеству Якоби } 
    $$
    [[x, y], z] + [[y, z], x] + [[z, x], y] = 0
    $$
\end{enumerate}


{ \bf Упражнение. } Отображение 
$$
[x, y] = xy - yx
$$ 
задает алгебру Ли.\\

{ \bf Пример. } \(\mathfrak{gl}(n, \mathbb{K})\) --- это алгебра Ли всех квадратных матриц размера \(n \times n\) над полем \(\mathbb{K}\), снабжённое скобкой Ли (коммутатором)
\[
[A, B] = AB - BA, \quad \forall A,B \in \mathfrak{gl}(n, \mathbb{K}).
\]


\hrulefill

\begin{center}
{\sc 2.1. Свободные алгебры Ли.}
\end{center}

{ \bf Определение. } Свободную алгебру Ли можно построить аналогично свободным группам. Пусть зафиксировано множество $\{x_i\}^n_{i=1}$, которое будем называть множеством образующих. Тогда свободная группа состоит из всевозможных применений скобок Ли парам элементам множества образующих и их порождающих.\\


{ \bf Определение. } Для группы Ли можно задать { \it структурированные константы} --- набор $\{f^k_{ij}\}^{(n, n, n)}_{(i, j, k) = (1, 1, 1)}$ для заданного базиса ${e_1, e_2, ..., e_n}$ как
$$
[e_i, e_j] = \sum_K f^k_{ij}e_k
$$

Аксиомы скобки Ли накладывают следующие соотношения на структурированные константы: $$\forall i,j,k$$
\begin{enumerate}
\item 
$
f_{ij}^k = -f_{ji}^k .
$

\item
$
\displaystyle\sum_{m=1}^n \left( f_{ij}^m f_{mk}^n + f_{jk}^m f_{mi}^n + f_{ki}^m f_{mj}^n \right) = 0.
$
\end{enumerate}


\hrulefill

\begin{center}
{\sc 2.2. Теорема Адо.}
\end{center}

Можно ли взять конечномерную алегбру Ли и вложить ее в трививальную?\\

{ \bf Теорема Адо. } Пусть $\mathfrak{g}$ -- конечномерная алгебра Ли над полем $\mathbb{K}$ характеристики 0. Тогда существует натуральное число $n$ и инъективный гомоморфизм алгебр Ли
\[
\varphi: \mathfrak{g} \hookrightarrow \mathfrak{gl}(n,\mathbb{K}).
\]

% Любую конечномерную алгебру можно рассматривать как подалгебру матриц, сильно упрощает жизнь.

% В алгебрах Ли ниче такого интересного нет. Вложение строить сложно.

% Алгебры необходимы для того, чтобы решать вопросы о группах, путем соотвествующего вложения.

\hrulefill

\begin{center}
{\sc 2.2. Связь категории групп с категориями алгебр Ли.}
\end{center}

Построим функтор из категории групп в категорию групп Ли.


Зафиксируем группу $G.$ Ее центральный нижний ряд имеет вид
$$
\gamma_{n+1}G \subseteq ... \subseteq \gamma_2G = [G, G] \subseteq G,
$$
причем $\gamma_n G / \gamma_{n+1}G$ --- абелева группа.

Определим алгебру Ли, как
$$
\mathbb{L} G \equiv \sum_{n \geq 1} \gamma_n G/ \gamma_{n+1}G,
$$
для которой скобка ли опредлена 
$$
\forall x \in \gamma_n G, y \in \gamma_m : 
$$
$$
[x\gamma_{n+1}G, y\gamma_{m+1}G] := [x, y] \gamma_{n+m+1}G 
$$
задает струткруру кольца Ли. 

\hrulefill

\begin{center}
{\sc 2.3. Свободное кольцо Ли.}
\end{center}

{ \bf Определение. } Пусть $A$ -- абелева группа. Тогда тензорная алгебра над $A$
\[
T(A) = \bigoplus_{m \geq 0} A^{\otimes m}.
\]

В $T(A)$ можно определить скобку Ли
\[
[a, b] := a \otimes b - b \otimes a.
\]


{ \bf Определение. } Свободное кольцо Ли --- подалгебра в $T(A)$:
\[
\mathbb{L}^{\mathrm{free}}(A) \subset T(A)^{\mathrm{Lie}}
\]
Элементы -- линейные комбинации коммутаторов:
\[
[a,b],\ [a,[b,c]],\ [[a,b],[c,d]],\ \ldots
\]

\hrulefill 

\begin{center}
{\sc 2.4. Теорема Магнуса-Вита.}
\end{center}

{ \bf Теорема Магнуса-Вита. } Пусть $F$ --- свободная группа. Тогда существует каноническое разложение:
\[
\mu: \mathbb{L}F \to \mathbb{L}^{free}(F_{ab}).
\]

{ \bf Определение. } $R_{ab}$ --- модуль соотношений. \\

{ \it "Это на самом деле комбинаторная теория групп."\\}

{ \bf Soon.}  Теорию Галуа обсудим позже.

% \hrulefill

% Старший модуль соотношений --- это эффективный модуль. Что значит эффектиный? Есть кольцо или группа если $R R_{ab}\times A \to A$
% $$
% ra = a \forall a \in A => r = 1
% $$


\hrulefill

Когда модуль соотношений является свободным 


Рассмотрим групповое кольцо. Она имеет идеал. I --- абелева подгруппа кольца R, такая что $\forall r \in R, i \in I$ тогда $ri \in I.$


Рассмотрим отображние $\varepsilon$ в Z
$$\sum n_iG_i : = \sum n_i$$

$\Delta(G)$ --- ядро $\varepsilon$ - агументеционный идеал. 

Квадрат идеала $I^2 = {\sum_i r_i(a_ib_i) : a_i \in I, b_i in I, r_i \in R}$

Рассмотрим отображение 
$$
G_{ab} \to \Delta(G) / \Delta(G)^2
$$
является изоморфизмом.

Доказательство. 

$(g_1-1)(g_2-1) \in \Delta(G)^2$ 

Таким образом
$$
(g_1-1)+(g_2-1) = g_1g_2-1 \pmod {\Delta(G)^2}
$$

\hrulefill

Пусть есть точная последовательность  
$$
R \to A \to G 
$$
--- копредставление группы $G$

$r$ --- идеал $(R-1)ZF = ker (ZF \to ZQ)$

$FR$ язык

$ (R-1)ZF = ker (ZF \to ZQ) \subset f = \Delta(F) \subset ZF$

Упражнение. $R_{ab}$ изоморфно $r/rf$


Мы пришли к вложению Магнуса. 
$$
r/rf \to f/rf \to f/r
$$

Есть вложения $R_{ab} \to ZG \to \Delta(G)$

Можно склеить точные последовательности.

{ \bf Определение. } Свободной резольвентой называется точная последовательность, каждый член которой свободен.

Рассмотрим функтор из категории $ZG-mod$ в $Ab$
$$
AG \equiv A/\{ga-a:a\in A, g\in G\}
$$
--- конинварианты.

$$\Delta(G)_G \cong \Delta(G)/\Delta(G)^2 \cong G_{ab}$$

На точную последовательность можно действовать функтором сохраняя условия точности.

\hrulefill

ГОМОЛОГИИ ГРУПП.

Пусть $Z$ --- тривиальный G-модуль и $... \to B_n \to ... \to B_1 \to Z$

Резольвета Блюмберга.

$\Im (d_{n+1})_G \subset Ker (d_n)_G$


$H_n(G) := ker(d_n)_G / Im (d_{n+1})_G$
--- гомологии, не зависят от выбора резольвенты.

$H_1(G) = G_{ab}$
$H_2(G) \cong R \cap [F, F] / [F, R]$ 


Теорема (Stallings). $F$ --- свободная тогда и только тогда, когда $\forall n \geq 2$ 
$$
H_n(F; A) = 0
$$
$\forall F-модуля A$.

$H_n(G; A)$ --- гомологии с коэффициентами.

}

\end{document}
